\documentclass[conference]{IEEEtran}
\IEEEoverridecommandlockouts
% The preceding line is only needed to identify funding in the first footnote. If that is unneeded, please comment it out.
\usepackage{cite}
\usepackage{amsmath,amssymb,amsfonts}
\usepackage{algorithmic}
\usepackage{graphicx}
\usepackage{textcomp}
\usepackage{xcolor}
%\usepackage{booktabs}
\usepackage{tabularray}
\usepackage{flushend}
\UseTblrLibrary{booktabs}
\def\BibTeX{{\rm B\kern-.05em{\sc i\kern-.025em b}\kern-.08em
    T\kern-.1667em\lower.7ex\hbox{E}\kern-.125emX}}
\begin{document}

\title{Analyzing shortwave propagation with a remote accessible software defined ham radio system}

\author{\IEEEauthorblockN{Gergely Vakulya}
\IEEEauthorblockA{\small \textit{Alba Regia Technical Faculty} \\
\textit{Óbuda University}\\
\textit{Székesfehérvár, Hungary}\\
\textit{vakulya.gergely@amk.uni-obuda.hu}
}
\and
\IEEEauthorblockN{Helga Anna Albert-Huszár}
\IEEEauthorblockA{\small \textit{Alba Regia Technical Faculty} \\
\textit{Óbuda University}\\
\textit{Székesfehérvár, Hungary}\\
\textit{albert.huszar.helga@amk.uni-obuda.hu}
}
}

\maketitle

\begin{abstract}
Ham radio has long been a foundational area of practice in electrical
  engineering. Advances in signal processing, particularly the advent of
  software-defined radio (SDR), have revolutionized the field, offering new
  possibilities and modes of operation. This paper introduces a system designed
  for long-term collection of shortwave propagation data, leveraging SDR
  technology. It also presents the analysis of the collected data,
  demonstrating the system's potential for advancing research in radio wave
  propagation.
\end{abstract}

\begin{IEEEkeywords}
software defined radio, shortwave propagation, data analysis, big data
\end{IEEEkeywords}

\section{Introduction}

Amateur radio, commonly known as ''ham radio'' offers a diverse range of
applications and goals for enthusiasts. However, regulatory authorities and
experts emphasize one critical objective: the study of radio wave propagation.
Propagation research through amateur radio plays a vital role in understanding
how radio signals travel, interact with the atmosphere, and behave under
various conditions.


The basic element of ham radio operation is the contact, in which the mandatory
parts are the exchange of the callsigns and the signal reports. Signal report
is the characterization of the received signal, which can contain the signal strength
and other properties (signal prurity, modulation quality, signal fluctuation etc.).
This information provides valuable data for studying the varying properties of radio
wave propagation.
Ham radio operators employ various modulation modes to facilitate voice, Morse
code, and data communication. In recent years, data-based contacts have gained
significant popularity, largely due to advancements in digital technology, such
as computers, sound cards, and specialized software. These highly efficient
digital modes have revolutionized ham radio by offering enhanced performance in
low-signal environments and expanding the possibilities for real-time
propagation studies.

FT8, or Franke-Taylor design, 8-Frequency Shift Keying, is a digital
communication mode developed in 2017 by Joe Taylor (K1JT) and Steven Franke
(K9AN). It has rapidly gained popularity within the amateur radio community due
to its exceptional efficiency, especially in low-signal and noisy environments.
FT8 utilizes small bandwidth (50 Hz) and transmits in 15-second intervals,
enabling operators to make reliable contacts even with weak signals, often
below the noise floor.

One of FT8's key features is its ability to perform automated,
computer-assisted communications, allowing for rapid exchanges of minimal data,
including callsigns, signal reports, and grid locations. Its structured and
streamlined communication process makes it an ideal tool for propagation
research, as it can quantify signal conditions across various bands in
real-time. The widespread use of FT8 has transformed amateur radio, expanding
the ability of operators to conduct long-distance contacts under challenging
propagation conditions with limited power and basic equipment.

\section{Architecture}

For this study, the 14 MHz (20-meter) band was chosen due to its consistent
activity levels throughout the day, making it ideal for observing and analyzing
propagation conditions over various times and regions. This band strikes a
balance between effective long-distance communication and manageable antenna
requirements, offering a practical setup for amateur radio research.

The antenna used in this research is a simple wire dipole, a straightforward
yet reliable design with a theoretical gain of 2.15 dBi. Its radiation pattern
is approximately omnidirectional, allowing for signal reception and
transmission in multiple directions without requiring complex equipment. This
configuration provides sufficient performance for studying propagation
characteristics, while maintaining a compact and accessible setup.

The mcHF transceiver was utilized in this study, covering the full range of
amateur radio bands from 1.8 MHz to 30 MHz. This versatile transceiver offers a
USB interface, enabling full control over key parameters such as frequency,
transmit/receive modes, and filter settings. Simultaneously, the USB connection
also serves as an audio interface, streamlining the integration with digital
communication software for data exchange and signal processing.

A significant advantage of the mcHF transceiver is its high third-order
intercept (IP3) point in the receiver stage. This high IP3 enhances the
receiver’s ability to manage weak signals, even in the presence of strong
nearby interference, making it particularly well-suited for experiments
involving low-power and low-signal conditions. Importantly, while the mcHF is
capable of transmission, the transmission function was disabled throughout this
experiment to focus exclusively on signal reception and propagation analysis.

The transceiver was connected to an Intel NUC DCCP847DYE mini-PC, equipped with
an Intel Celeron 847E CPU running at 1.10GHz, 8 GB of RAM, and 128 GB of NVMe
SSD storage. Despite its relatively modest specifications, this
resource-constrained setup provides sufficient computing power to effectively
control the radio transceiver and execute signal processing algorithms, all
while maintaining low energy consumption. Additionally, the PC features a power
management setting that automatically powers the device on when electricity is
restored after an outage.

The PC runs Debian Linux 12.5 in headless mode, i.e. only remote access is
possible. A remote shell access is configured via SSH (Secure Shell).
As the signal processing
software (WSJT-X, see later) requires graphical mode a VNC (Virtual Network Computing)
\cite{richardson1998}
session is configured
to start automatically during startup. This way the desktop can be accessed
through a VNCviewer software, e.g. RealVNC. 

VPN

WSJT-X

Log files


\section{Results}

\begin{figure}[htbp]
	\centering
	\includegraphics[width=.5\textwidth]{fig/map.png}
	\caption{Map of 4-character Maidenhead grid squares, where CQ messages were received from.
  More red colors denote more active areas.}
	\label{fig-dummy}
\end{figure}

\section{Summary}


\bibliographystyle{IEEEtran}
\bibliography{IEEEabrv,references}

\end{document}
