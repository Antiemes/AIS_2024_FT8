\documentclass[conference]{IEEEtran}
\IEEEoverridecommandlockouts
% The preceding line is only needed to identify funding in the first footnote. If that is unneeded, please comment it out.
\usepackage{cite}
\usepackage{amsmath,amssymb,amsfonts}
\usepackage{algorithmic}
\usepackage{graphicx}
\usepackage{textcomp}
\usepackage{xcolor}
%\usepackage{booktabs}
\usepackage{tabularray}
\usepackage{flushend}
\UseTblrLibrary{booktabs}
\def\BibTeX{{\rm B\kern-.05em{\sc i\kern-.025em b}\kern-.08em
    T\kern-.1667em\lower.7ex\hbox{E}\kern-.125emX}}
\begin{document}

\title{Analyzing shortwave propagation with a remote accessible ham radio system}

\author{\IEEEauthorblockN{Gergely Vakulya}
\IEEEauthorblockA{\small \textit{Alba Regia Technical Faculty} \\
\textit{Óbuda University}\\
\textit{Székesfehérvár, Hungary}\\
\textit{vakulya.gergely@amk.uni-obuda.hu}
}
\and
\IEEEauthorblockN{Helga Anna Albert-Huszár}
\IEEEauthorblockA{\small \textit{Alba Regia Technical Faculty} \\
\textit{Óbuda University}\\
\textit{Székesfehérvár, Hungary}\\
\textit{albert.huszar.helga@amk.uni-obuda.hu}
}
}

\maketitle

\begin{abstract}
Ham radio has long been a foundational area of practice in electrical
  engineering. Advances in signal processing, particularly the advent of
  software-defined radio (SDR), have revolutionized the field, offering new
  possibilities and modes of operation. This paper introduces a system designed
  for long-term collection of shortwave propagation data, leveraging SDR
  technology. It also presents the analysis of the collected data,
  demonstrating the system's potential for advancing research in radio wave
  propagation.
\end{abstract}

\begin{IEEEkeywords}
software defined radio, shortwave propagation, data analysis
\end{IEEEkeywords}

\section{Introduction}

\cite{ga-vrp}

\section{Related work}

\begin{figure}[htbp]
	\centering
	\includegraphics[width=0.35\textwidth]{fig/dummy.png}
	\caption{Dummy image}
	\label{fig-dummy}
\end{figure}

\section{Summary}


\bibliographystyle{IEEEtran}
\bibliography{IEEEabrv,references}

\end{document}
