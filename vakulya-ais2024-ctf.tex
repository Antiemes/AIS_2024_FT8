\documentclass[conference]{IEEEtran}
\IEEEoverridecommandlockouts
% The preceding line is only needed to identify funding in the first footnote. If that is unneeded, please comment it out.
\usepackage{cite}
\usepackage{amsmath,amssymb,amsfonts}
\usepackage{algorithmic}
\usepackage{graphicx}
\usepackage{textcomp}
\usepackage{xcolor}
%\usepackage{booktabs}
\usepackage{tabularray}
\usepackage{flushend}
\UseTblrLibrary{booktabs}
\def\BibTeX{{\rm B\kern-.05em{\sc i\kern-.025em b}\kern-.08em
    T\kern-.1667em\lower.7ex\hbox{E}\kern-.125emX}}
\begin{document}

\title{Gamifying Cybersecurity:\\The CTF Challenges}

\author{\IEEEauthorblockN{Gergely Vakulya}
\IEEEauthorblockA{\small \textit{Alba Regia Technical Faculty} \\
\textit{Óbuda University}\\
\textit{Székesfehérvár, Hungary}\\
\textit{vakulya.gergely@amk.uni-obuda.hu}
}
\and
\IEEEauthorblockN{Helga Anna Albert-Huszár}
\IEEEauthorblockA{\small \textit{Alba Regia Technical Faculty} \\
\textit{Óbuda University}\\
\textit{Székesfehérvár, Hungary}\\
\textit{albert.huszar.helga@amk.uni-obuda.hu}
}
}

\maketitle

\begin{abstract}
Cybersecurity is becoming increasingly critical in today's digital landscape,
  yet it remains challenging to enter the field due to its inherently complex
  nature, requiring expertise across a wide range of specialized areas. This
  paper explores the use of Capture the Flag (CTF) challenges as a gamified
  method for enhancing cybersecurity education and skill development. By
  integrating real-world scenarios into a competitive format, CTF challenges
  encourage participants to apply their knowledge in dynamic, problem-solving
  environments. We analyze various CTF challenge categories, and discuss their
  effectiveness in fostering critical cybersecurity skills.
\end{abstract}

\begin{IEEEkeywords}
cybersecurity, linux, forensics, cryptography
\end{IEEEkeywords}

\section{Introduction}

\cite{ga-vrp}

\section{Related work}

\begin{figure}[htbp]
	\centering
	\includegraphics[width=0.35\textwidth]{fig/dummy.png}
	\caption{Dummy image}
	\label{fig-dummy}
\end{figure}

\section{Summary}


\bibliographystyle{IEEEtran}
\bibliography{IEEEabrv,references}

\end{document}
